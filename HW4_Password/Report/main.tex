%%%%%%%%%%%%%%%%%%%%%%%%%%%%%%%%%%%%%%%%%

% Lachaise Assignment
% LaTeX Template
% Version 1.0 (26/6/2018)
%
% This template originates from:
% http://www.LaTeXTemplates.com
%
% Authors:
% Marion Lachaise & François Févotte
% Vel (vel@LaTeXTemplates.com)
%
% License:
% CC BY-NC-SA 3.0 (http://creativecommons.org/licenses/by-nc-sa/3.0/)
% 
%%%%%%%%%%%%%%%%%%%%%%%%%%%%%%%%%%%%%%%%%

%----------------------------------------------------------------------------------------
%	PACKAGES AND OTHER DOCUMENT CONFIGURATIONS
%----------------------------------------------------------------------------------------

\documentclass[UTF8]{article}

\input{structure.tex} % Include the file specifying the document structure and custom commands

%----------------------------------------------------------------------------------------
%	ASSIGNMENT INFORMATION
%----------------------------------------------------------------------------------------

\title{数字逻辑实验 \\ 串行密码锁设计} % Title of the assignment

\author{姓名:任一  \\学号:2018011423\\ \texttt{ry18@mails.tsinghua.edu.cn}} % Author name and email address

\date{\today} % University, school and/or department name(s) and a date

%----------------------------------------------------------------------------------------
\lstset{
    % backgroundcolor=\color{red!50!green!50!blue!50},%代码块背景色为浅灰色
    rulesepcolor= \color{gray}, %代码块边框颜色
    breaklines=true,  %代码过长则换行
    numbers=left, %行号在左侧显示
    numberstyle= \small,%行号字体
    keywordstyle= \color{blue},%关键字颜色
    commentstyle=\color{gray}, %注释颜色
    frame=shadowbox%用方框框住代码块
}

\begin{document}

\maketitle % Print the title
\begin{center}
    \begin{tabular}{l  r}
    \hline

        \multicolumn{2}{c}{实验环境} \\ \hline
        操作系统: & Windows10家庭版 18362.72 \\ \hline% Date the experiment was performed
        QuartusII版本: & Quartus II 13.0 sp1 \\ \hline% Partner names

        ModelSim版本: & Modelsim SE-64 10.7 \\ \hline% Instructor/supervisor

    \end{tabular}
\end{center}
\newpage


\section{实验概述}
\subsection{实验思路}
在本实验中,我使用状态机,实现了串行密码锁的设计,具有密码设置和输入密码两种状态,并且能通过LED灯提示
输入的密码是否正确。

具体来说,我的状态有6个。分别是start, in1, in2, in3, check1, check2, check3, final.
当模式为设置密码模式时,状态转移为start->in1->in2->in3->final.
当模式为检查密码模式时,若输入正确,状态转移为start->check1->check2->check3->final并亮起表示输入正确的灯.
若输入错误,则状态回到start并亮起表示输入错误的灯。

\subsection{文件说明}
Password文件夹下是串行密码锁的工程文件和代码,
JieLabVideo文件夹下是在JieLab在线平台上实现的串行密码锁测试的录屏文件。


\section{实验结果}
具体实验结果见JieLabVideo文件夹下的演示视频。仿真时的一张截图如下:

\begin{figure}[h]
    \label{md}
    \centering
        \includegraphics[width=0.8\textwidth]{p1.png}
        \caption{在线JieLabs仿真截图}
    \end{figure}

\section{实验总结}
在本次实验中,我尝试了状态机的设计,为数字逻辑设计课程中的设计
打下了坚实的理论基础,感谢助教和老师的指教!


\bibliographystyle{plain}
\bibliography{ref} %这里的这个ref就是对文件ref.bib的引用

\end{document}
