%%%%%%%%%%%%%%%%%%%%%%%%%%%%%%%%%%%%%%%%%

% Lachaise Assignment
% LaTeX Template
% Version 1.0 (26/6/2018)
%
% This template originates from:
% http://www.LaTeXTemplates.com
%
% Authors:
% Marion Lachaise & François Févotte
% Vel (vel@LaTeXTemplates.com)
%
% License:
% CC BY-NC-SA 3.0 (http://creativecommons.org/licenses/by-nc-sa/3.0/)
% 
%%%%%%%%%%%%%%%%%%%%%%%%%%%%%%%%%%%%%%%%%

%----------------------------------------------------------------------------------------
%	PACKAGES AND OTHER DOCUMENT CONFIGURATIONS
%----------------------------------------------------------------------------------------

\documentclass[UTF8]{article}

\input{structure.tex} % Include the file specifying the document structure and custom commands

%----------------------------------------------------------------------------------------
%	ASSIGNMENT INFORMATION
%----------------------------------------------------------------------------------------

\title{数字逻辑实验 \\ 逐次进位加法器与超前进位加法器设计} % Title of the assignment

\author{姓名:任一  \\学号:2018011423\\ \texttt{ry18@mails.tsinghua.edu.cn}} % Author name and email address

\date{\today} % University, school and/or department name(s) and a date

%----------------------------------------------------------------------------------------
\lstset{
    % backgroundcolor=\color{red!50!green!50!blue!50},%代码块背景色为浅灰色
    rulesepcolor= \color{gray}, %代码块边框颜色
    breaklines=true,  %代码过长则换行
    numbers=left, %行号在左侧显示
    numberstyle= \small,%行号字体
    keywordstyle= \color{blue},%关键字颜色
    commentstyle=\color{gray}, %注释颜色
    frame=shadowbox%用方框框住代码块
}

\begin{document}

\maketitle % Print the title
\begin{center}
    \begin{tabular}{l  r}
    \hline

        \multicolumn{2}{c}{实验环境} \\ \hline
        操作系统: & Windows10家庭版 18362.72 \\ \hline% Date the experiment was performed
        QuartusII版本: & Quartus II 13.0 sp1 \\ \hline% Partner names

        ModelSim版本: & Modelsim SE-64 10.7 \\ \hline% Instructor/supervisor

    \end{tabular}
\end{center}
\newpage


\section{实验概述}
\subsection{实验思路}
在本实验中,我实现了逐次进位加法器与超前进位加法器。
实现方法主要是通过元件例化,先设计半加器,利用半加器设计全加器,再用
全加器设计逐次进位加法器与超前进位加法器。这样的逐层设计方法,有利于代码复用,
提高了开发效率,降低了开发难度。

\subsection{文件说明}
Serial4FullAdder文件夹下是逐次进位加法器的工程文件和代码,
CarryLookAheadAdder文件夹中是超前进位加法器的工程文件和代码,
这两个文件夹下的ModelSimTestBench文件夹中是ModelSim模拟的工程文件。
SimulationScreenshots文件夹中是使用Testbench在ModelSim中进行仿真的截图。
JiebaScreenShots文件夹下是上述两种加法器在Jieba数电平台上的仿真截图。

\section{实验结果}
\subsection{Testbench仿真}
\begin{figure}[h]
    \label{md}
    \centering
        \includegraphics[width=0.8\textwidth]{SFA.png}
        \caption{逐次进位加法器仿真测试截图}
    \end{figure}
    \begin{figure}[h]
        \label{md}
        \centering
            \includegraphics[width=0.8\textwidth]{CLA.png}
            \caption{超前进位加法器仿真测试截图}
        \end{figure}

由仿真截图可以看出,我设计的逐次进位加法器与超前进位加法器很好地完成了
加法和进位的任务。
\clearpage

\subsection{RTL Viewer}
\begin{figure}[h]
    \label{md}
    \centering
        \includegraphics[width=0.8\textwidth]{Serial4FullAdder.png}
        \caption{逐次进位加法器在CPLD中的电路图}
    \end{figure}
    \begin{figure}[h]
        \label{md}
        \centering
            \includegraphics[width=0.8\textwidth]{CarryLookAheadAdder.png}
            \caption{超前进位加法器在CPLD中的电路图}
        \end{figure}
通过QuartusII软件中的RTL Viewer功能,我得到了这两种加法器啊在CPLD中的电路图。
可以清晰地看出,逐次进位加法器的电路比逐次进位加法器的电路要简洁一些。

\clearpage
\subsection{Jielab在线测试}
\begin{figure}[h]
    \label{md}
    \centering
        \includegraphics[width=0.8\textwidth]{ZC.png}
        \caption{逐次进位加法器仿真测试截图}
    \end{figure}

\begin{figure}[h]
    \label{md}
    \centering
        \includegraphics[width=0.8\textwidth]{CQ.png}
        \caption{超前进位加法器仿真测试截图}
    \end{figure}
从图中可以清晰看到,逐次进位加法器实现了$0100+1111=10011$, 超前进位加法器实现了$1101+1001=10110$的任务。

\clearpage

\section{实验总结}
在本次实验中,我感受到了元件例化所带来的便利,提高了VHDL编程能力,
为后续项目的设计打下了基础。感谢老师和助教的悉心指导!


\bibliographystyle{plain}
\bibliography{ref} %这里的这个ref就是对文件ref.bib的引用

\end{document}
