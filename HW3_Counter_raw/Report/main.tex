%%%%%%%%%%%%%%%%%%%%%%%%%%%%%%%%%%%%%%%%%

% Lachaise Assignment
% LaTeX Template
% Version 1.0 (26/6/2018)
%
% This template originates from:
% http://www.LaTeXTemplates.com
%
% Authors:
% Marion Lachaise & François Févotte
% Vel (vel@LaTeXTemplates.com)
%
% License:
% CC BY-NC-SA 3.0 (http://creativecommons.org/licenses/by-nc-sa/3.0/)
% 
%%%%%%%%%%%%%%%%%%%%%%%%%%%%%%%%%%%%%%%%%

%----------------------------------------------------------------------------------------
%	PACKAGES AND OTHER DOCUMENT CONFIGURATIONS
%----------------------------------------------------------------------------------------

\documentclass[UTF8]{article}

\input{structure.tex} % Include the file specifying the document structure and custom commands

%----------------------------------------------------------------------------------------
%	ASSIGNMENT INFORMATION
%----------------------------------------------------------------------------------------

\title{数字逻辑实验 \\ 计数器设计} % Title of the assignment

\author{姓名:任一  \\学号:2018011423\\ \texttt{ry18@mails.tsinghua.edu.cn}} % Author name and email address

\date{\today} % University, school and/or department name(s) and a date

%----------------------------------------------------------------------------------------
\lstset{
    % backgroundcolor=\color{red!50!green!50!blue!50},%代码块背景色为浅灰色
    rulesepcolor= \color{gray}, %代码块边框颜色
    breaklines=true,  %代码过长则换行
    numbers=left, %行号在左侧显示
    numberstyle= \small,%行号字体
    keywordstyle= \color{blue},%关键字颜色
    commentstyle=\color{gray}, %注释颜色
    frame=shadowbox%用方框框住代码块
}

\begin{document}

\maketitle % Print the title
\begin{center}
    \begin{tabular}{l  r}
    \hline

        \multicolumn{2}{c}{实验环境} \\ \hline
        操作系统: & Windows10家庭版 18362.72 \\ \hline% Date the experiment was performed
        QuartusII版本: & Quartus II 13.0 sp1 \\ \hline% Partner names

        ModelSim版本: & Modelsim SE-64 10.7 \\ \hline% Instructor/supervisor

    \end{tabular}
\end{center}
\newpage


\section{实验概述}
\subsection{实验思路}
在本实验中,我实现了计数器设计,并采用了结构化的设计方式。
设计思路如下:个位的计数采用D触发器实现十进制计数器。
十位的计数采用D触发器实现6进制计数器。
同时,D触发器还具有异步清零功能,使得Reset功能得以实现。

对于手动点击的计数器,只需要将CLK信号接到开关上即可使用。
对于自动的计数器(即秒表),需要将1MHz的时钟信号做分频,得到1Hz的时钟信号,从而得以实现秒表。


\subsection{文件说明}
Counter文件夹下是计数器的工程文件和代码(包括TestBench文件),
JieLabVideo文件夹下是上述手动计数器和自动计数器在JieLab实验平台上仿真的录频文件。


\section{实验结果}
本实验在本地的TestBench测试和线上JieLab平台上的测试都较为顺利,体现了本计数器设计的稳定性和正确性。
\subsection{Testbench仿真截图}
\begin{figure}[h]
    \label{md}
    \centering
        \includegraphics[width=0.8\textwidth]{sc1.png}
        \caption{计数器TestBench仿真}
    \end{figure}
    \begin{figure}[h]
        \label{md}
        \centering
            \includegraphics[width=0.8\textwidth]{sc2.png}
            \caption{计数器TestBench仿真}
        \end{figure}

\subsection{Jielab在线测试}
具体内容在JieLabVideo文件夹下,由视频可以看出本计数器很好地完成了实验要求。

\section{实验总结}
在本次实验中,我体验了结构化设计的方式,感谢老师和助教的悉心指教!


\bibliographystyle{plain}
\bibliography{ref} %这里的这个ref就是对文件ref.bib的引用

\end{document}
